\documentclass[a4paper]{article}

%% Language and font encodings
\usepackage[english]{babel}
\usepackage{amsthm}
\usepackage[utf8x]{inputenc}
\usepackage[T1]{fontenc}
\usepackage{listings}
\usepackage{algpseudocode, algorithm}
\usepackage{mathtools}
\usepackage{enumerate}
\usepackage{tabu}

%% Sets page size and margins
\usepackage[a4paper,top=3cm,bottom=2cm,left=3cm,right=3cm,marginparwidth=1.75cm]{geometry}

%% Useful packages
\usepackage{amsmath}
\usepackage{graphicx}
\usepackage[colorinlistoftodos]{todonotes}
\usepackage[colorlinks=true, allcolors=blue]{hyperref}

\newenvironment{solution}{\begin{proof}[\textnormal{\textbf{Solution}}]}{\end{proof}}
\newenvironment{exercise}[1]{\begin{proof}[\textnormal{\textbf{Exercise #1:}}]\phantom{\qedhere}}{\end{proof}}
\newenvironment{lemma}{\begin{proof}[\textnormal{\textbf{Lemma}}]\phantom{\qedhere}}{\end{proof}}
\newenvironment{definition}[1]{\begin{proof}[\textnormal{\textbf{Definition: #1}}]\mbox{}\\\phantom{\qedhere}}{\end{proof}}

\theoremstyle{definition}
\newtheorem*{thm}{Theorem}

\begin{document}
\begin{titlepage}\pagenumbering{gobble}
	\centering
	{\scshape\LARGE University of Houston\par}
	\vspace{1cm}
	{\scshape\Large Final-Exam Review \par}
	\vspace{1.5cm}
	{\huge\bfseries COSC 3320 \par}
	{\huge\bfseries Algorithms and Data Structures\par}
	\vspace{0.5cm}
	{\large\bfseries Gopal Pandurangan\par}
	\vspace{2cm}
	\vfill

% Bottom of the page
\end{titlepage}
\vspace*{\fill}\begin{center}{\Huge This page intentionally left blank.}\end{center}\vspace*{\fill}\thispagestyle{empty}\clearpage
\pagenumbering{arabic}
\setcounter{section}{-1}
\section{Previous Concepts}
Concepts seen previously can be found in the mid-term review, located \href{https://github.com/KalHourani/algorithms-and-data-structures-mid-term-review/raw/master/mid-term-review.pdf}{here}.
\setcounter{section}{0}
\section{Greedy Algorithms}

\section{Heaps}

\section{Graphs}
\subsection{Representing a Graph}
\subsection{Depth-First Search}
\subsection{Breadth-First Search}

\section{Graph Algorithms}
\subsection{Minimum Spanning Tree}
\subsection{Shortest Paths}
\end{document}
